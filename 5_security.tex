\section{Protocol Security}
\label{sec:security}

Yao's protocol is designed to provide \ac{SFE} against \emph{semi-honest} adversaries. These security guarantees do not carry over against \emph{malicious} adversaries though.  This is a serious shortcoming for being able to make the protocol practical; there are relatively few real-world scenarios where you do not trust the other party to see your inputs to a function, but do trust them to forgo the opportunity to discover those same inputs by deviating from the protocol.

Much work has been conducted to extend Yao's protocol to be secure against \emph{malicious} adversaries.  This work can generally be classified into three areas, 1) creating \emph{1-out-of-2 \ac{OT}} protocols that are secure against \emph{malicious} adversaries, 2) ensuring that the circuit constructing party correctly constructs the garbled circuit, and 3) that the circuit executing party returns provides the value output by the circuit to the other party.

\subsection{Securing the \ac{OT} Protocol}

The \emph{1-out-of-2 \ac{OT}} protocol described in the section \ref{sec:ot} is trivially vulnerable in the \emph{malicious} case.  Instead of generating $((k^{pub}_b, k^{pri}_b), (k^\bot_{b-1}, \bot))$, \ac{P2} could easily generate two valid public / private key pairs, allowing her to recover both values sent by \ac{P1}. Applied to Yao's protocol, this would allow \ac{P2} to learn both the garbled versions of the 0 and 1 values for all of her input bits. \ac{P2} having these additional keys would allow \ac{P2} to decrypt additional values throughout the circuit garbled gate, violating the \emph{privacy} requirement of \ac{SFE}.  Others have detailed several additional ways that using an insecure-in-the-\emph{malicious}-case, \ac{OT} protocol can be exploited by an attacker\cite{kiraz2006protocol}.

As previously discussed, \ac{OT} is a distinct, though related, field to \ac{SFE} in general and Yao's protocol in particular.  As such, this section does not attempt to assess the state of the art of in the field of \ac{OT}.  A variety of other approaches to \emph{malicious}-case secure \emph{1-out-of-2 \ac{OT}} protocols exist\cite{naor2001efficient, kiraz2006protocol, goldreich1987play}, each with their own tradeoffs, computation cost and underlying security assumptions.  The below protocol\cite{bellare1990non}\footnote{This protocol is a slightly modified version of the protocol presented in \cite{bellare1990non}, to incorporate a change suggested by \cite{naor2001efficient} to remove the reliance on a external zero knowledge proof or other out-side-the-protocol source for \emph{C}.} is included to show that efficient \emph{1-out-of-2 \ac{OT}} is possible, and that researchers have used it and equivalent \ac{OT} protocols to make Yao's \ac{GCP} secure in the \emph{malicious} case.

\begin{algorithm}[H]
    \floatname{algorithm}{Protocol}
    \caption{Malicious-Secure 1-out-of-2 Oblivious Transfer}
    \label{alg:otmalicious}
    \begin{algorithmic}[1]
        \STATE \ac{P1} has a set of two strings, $S = \{s_0, s_1\}$.
        \STATE \ac{P1} (sender) and \ac{P2} (receiver) agree on some $q$ and $g$ such that $g$ is a generator for $\mathbb{Z}^*_q$.
        \STATE \ac{P1} selects a random $C$ from $\mathbb{Z}^*_q$, or more generally such that \ac{P2} does not know the discrete log of $C$ in $\mathbb{Z}^*_q$.
        \STATE \ac{P2} selects $i \in \{0, 1\}$ corresponding to whether \ac{P2} wants $s_0$ or $s_1$. \ac{P2} also selects a random $0 \leq x_i \leq q-2$.
        \STATE \ac{P2} sets $\beta_i = g^{x_i}$ and $\beta_{i-1} = C \bullet (g^{x_i})^-1$. $(\beta_0, \beta_1)$ and $(i, x_i)$ form \ac{P1} public and private keys, respectively.

        \STATE \ac{P1} checks the validity of \ac{P2}'s public keys by verifying that $\beta_0 \bullet \beta_1 = C$.  If not, \ac{P1} aborts.

        \STATE \ac{P1} selects $y_0, y_1$ such that $0 \leq y_0, y_1 \leq q-2$, and sends \ac{P2} $a_0 = g^{y_0}, a_1 = g^{y_1}$.
        \STATE \ac{P1} also generates $z_0 = \beta^{y_0}_0, z_1 = \beta^{y_1}_1$ and sends \ac{P2} $r_0 = s_0 \oplus z_0$ and $r_1 = s_1 \oplus z_1$.
        \STATE \ac{P2} computes $z_i = a^{xi}_i$ and then receives $s_i$ by computing $s_i = z_i \oplus r_i$.
    \end{algorithmic}
\end{algorithm}

Step 5 is not in mentioned in the original\cite{bellare1990non} statement of the protocol, but was added by the author to better explain how the protocol works. Here, note that \ac{P1} checks that $\beta_0 \bullet \beta_1 = C$ to prevent \ac{P2} from being able to decrypt under both $\beta_0$ and $\beta_1$, to force \ac{P2} to choose one or the other. As long as the assumption that \ac{P2} does not know the discrete log of $C$ holds, then it follows that \ac{P2} cannot know the discrete log of both $\beta_0$ and $\beta_1$.

\subsection{Securing Circuit Construction}

A second way a malicious adversary could exploit Yao's protocol to learn information about the other party's input is by \ac{P1} creating and garbling a circuit for a function other than the function expected by \ac{P2}. Trivially, \ac{P1} could send \ac{P2} a garbled circuit couple simply send \ac{P2}'s input to \ac{P1}, or leak information about $i_{P2}$ in some other manner not known by \ac{P2}.  It is therefor necessary for \ac{P2} to ensure that the garbled circuit she computes is actually a garbled representation of the expected function.

\subsubsection{Zero-Knowledge Proofs}

Two different general strategies for achieving this assurance have been promoted.  The first approach is that \ac{P1} generates a zero knowledge proof of the garbled circuit's correctness, and then sends this proof to \ac{P2} along with the garbled circuit and \ac{P1}'s garbled inputs\cite{goldreich1987play, goldreich2009foundations}.

This zero-knowledge strategy dates back to earlier in this history of Yao's protocol, when the protocol was treated as proof that \ac{SFE} was possible, instead of as a practical tool for actually achieving \ac{SFE}. More recent, implementation-focused work on Yao's protocol has treated the zero-knowledge proof approaches as too expensive for practical use\cite{lindell2007efficient, mohassel2006efficiency, malkhi2004fairplay}.

\subsubsection{Cut-and-Choose}

Instead, recent work on Yao's protocol has focused on a \emph{cut-and-choose} strategy for securing circuit construction\cite{malkhi2004fairplay}. Work on this approach has developed in an arms-race fashion, with proposals being made, other researchers revealing shortcomings in the given strategy, and a new strategy being developed to address the given short coming. Several rounds of this propose-attack-revise cycle are discussed below, to give context and meaning to each proposed improvement.\\[.5em]

\noindent\textbf{Standard Cut-and-Choose}

Under this approach, \ac{P1} constructs $m$ versions of the circuit, each structured the same but garbled so that the keys for each gate in the circuit are all unique. \ac{P1} does the same for his inputs to each of the garbled circuits. Additionally, \ac{P1} generates a commitment for each of his garbled inputs, which for simplicity can be represented by a simple hash function.\footnote{though several more secure methods of committing are mentioned in \cite{lindell2007efficient}, a simple hash function is used here to simplify the description of the cut-and-choose approach.  A more secure approach is described here \cite{halevi1996practical}},

\ac{P1} then sends each of these pairs of garbled circuits and associated input commitments to \ac{P2}, who selects $m-1$ versions of the circuit to verify.  \ac{P1} de-garbles each of the $m-1$ selected circuits, so that \ac{P2} can see the underlying circuit with the obscured boolean values in each gates' computation table.  \ac{P2} can then verify that each of the revealed circuits is constructed correctly.

If everything looks correct to \ac{P2} she will continue with the computation by receiving \ac{P1} garbled inputs, checking that they match the previously sent commitment (again, most simply thought of as checking that the hash of the received garbled inputs matches the previously sent hash), and then proceed with Yao's protocol as normal. This reduces the chances of \ac{P1} tricking \ac{P2} into computing a corrupted circuit to $1/m$. This protocol is provided more formally in Protocol \ref{alg:cutandchoose-basic}

\begin{algorithm}[H]
    \floatname{algorithm}{Protocol}
    \caption{Securing Circuit Construction With Cut-and-Choose}
    \label{alg:otmalicious}
    \begin{algorithmic}[1]
        \STATE \ac{P1} generates $m$ garbled versions of the circuit $c$, along with a corresponding garbled version of his input, called $X_i$ for $0 \leq i < m$.
        \STATE \ac{P1} uses hash function $H$ to generate commitments to each garbled input, $COMMIT_i = H(X_i) for 0 \leq i < m$.

        \STATE \ac{P1} sends \ac{P2} $m$ garbled circuits and $COMMIT$ such that $|COMMIT| = m$.
        \STATE \ac{P2} selects $0 \leq j < m$ and \ac{P1} un-garbles all circuits except $j$.
        \STATE \ac{P2} inspects all $m-1$ circuits to check that they are correctly formed. If not, \ac{P2} aborts.

        \STATE \ac{P2} receives \ac{P1}'s garbled inputs to circuit $j$ and confirms that \ac{P1} did not change his inputs by verifying $COMMIT_j = H(X_j)$. If not, \ac{P2} aborts.

        \STATE Otherwise, \ac{P2} receives the continues with Yao's protocol as normal.
    \end{algorithmic}
\end{algorithm}


\noindent\textbf{Further Securing Cut-and-Choose}

The above protocol gives \ac{P1} a $1/m$ chance of deceiving \ac{P2} into computing a corrupted version of the circuit.  However, the cut-and-choose technique can be improved, and the chances of \ac{P1} succeeding in this deception reduced exponentially by having \ac{P2} have \ac{P1} reveal only half of the $m$ circuits.  \ac{P2} then computes the remaining $m/2$ circuits and outputs the majority result of the remaining circuits.



% 6.3. Further Securing Circut Selection using Cut-and-Choose
%     * Lindell, Pinkas, An Efficient Protocol For Secure Two-Party Computation in the Presence of Malicious Adversaries [2007]
%     - Cut and choose is a step in the right direction, but reveals its own set of problems
%         - probabiltiy of Bob succedding in cheating can be reduced by Alice examining 1/2 of circuits, and then evaluating the remainder and taking the majority result, reducing the chance of Bob succeeding to 2^−0.311s for even modest values of s (ex 132).  This is because, in order to succeed, Bob must be lucky enough that none of his corrupted circuits appear in the 1/2 selected to be ungarbled, and more than half the remanining circuits must be corrupted.
%             * Yehuda Lindell, "Secure Two-Party Computation via Cut-and-Choose Oblivious Transfer" [2011]
%         - Securing this reveal-half method requires taking the majority result of the remaining, executed circuits.  The alternative, taking the result only if all circuits match in execution, allows for another exploit.
%             * Lindell, Pinkas, An Efficient Protocol For Secure Two-Party Computation in the Presence of Malicious Adversaries [2007]

