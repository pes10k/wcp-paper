\section{Protocol Security}
\label{sec:security}

Yao's protocol is designed to provide \ac{SFE} against \emph{semi-honest} adversaries. These security guarantees do not carry over against \emph{malicious} adversaries though.  This is a serious shortcoming for being able to make the protocol practical; there are relatively few real-world scenarios where you do not trust the other party to see your inputs to a function, but do trust them to forgo the opportunity to discover those same inputs by deviating from the protocol.

Much work has been conducted to extend Yao's protocol to be secure against \emph{malicious} adversaries.  This work can generally be classified into three areas, 1) creating \emph{1-out-of-2 \ac{OT}} protocols that are secure against \emph{malicious} adversaries, 2) ensuring that the circuit constructing party correctly constructs the garbled circuit, and 3) that the circuit executing party returns provides the value output by the circuit to the other party.

\subsection{Securing the \ac{OT} Protocol}

The \emph{1-out-of-2 \ac{OT}} protocol described in the section \ref{sec:ot} is trivially vulnerable in the \emph{malicious} case.  Instead of generating $((k^{pub}_b, k^{pri}_b), (k^\bot_{b-1}, \bot))$, \ac{P2} could easily generate two valid public / private key pairs, allowing her to recover both values sent by \ac{P1}. Applied to Yao's protocol, this would allow \ac{P2} to learn both the garbled versions of the 0 and 1 values for all of her input bits. \ac{P2} having these additional keys would allow \ac{P2} to decrypt additional values throughout the circuit garbled gate, violating the \emph{privacy} requirement of \ac{SFE}.  Others have detailed several additional ways that using an insecure-in-the-\emph{malicious}-case, \ac{OT} protocol can be exploited by an attacker\cite{kiraz2006protocol}.

As previously discussed, \ac{OT} is a distinct, though related, field to \ac{SFE} in general and Yao's protocol in particular.  As such, this section does not attempt to assess the state of the art of in the field of \ac{OT}.  A variety of other approaches to \emph{malicious}-case secure \emph{1-out-of-2 \ac{OT}} protocols exist\cite{naor2001efficient, kiraz2006protocol, goldreich1987play}, each with their own tradeoffs, computation cost and underlying security assumptions.  The below protocol\cite{bellare1990non}\footnote{This protocol is a slightly modified version of the protocol presented in \cite{bellare1990non}, to incorporate a change suggested by \cite{naor2001efficient} to remove the reliance on a external zero knowledge proof or other out-side-the-protocol source for \emph{C}.} is included to show that efficient \emph{1-out-of-2 \ac{OT}} is possible, and that researchers have used it and equivalent \ac{OT} protocols to make Yao's \ac{GCP} secure in the \emph{malicious} case.

\begin{algorithm}[H]
    \floatname{algorithm}{Protocol}
    \caption{Malicious-Secure 1-out-of-2 Oblivious Transfer}
    \label{alg:otmalicious}
    \begin{algorithmic}[1]
        \STATE \ac{P1} has a set of two strings, $S = \{s_0, s_1\}$.
        \STATE \ac{P1} (sender) and \ac{P2} (receiver) agree on some $q$ and $g$ such that $g$ is a generator for $\mathbb{Z}^*_q$.
        \STATE \ac{P1} selects a random $C$ from $\mathbb{Z}^*_q$, or more generally such that \ac{P2} does not know the discrete log of $C$ in $\mathbb{Z}^*_q$.
        \STATE \ac{P2} selects $i \in \{0, 1\}$ corresponding to whether \ac{P2} wants $s_0$ or $s_1$. \ac{P2} also selects a random $0 \leq x_i \leq q-2$.
        \STATE \ac{P2} sets $\beta_i = g^{x_i}$ and $\beta_{i-1} = C \bullet (g^{x_i})^-1$. $(\beta_0, \beta_1)$ and $(i, x_i)$ form \ac{P1} public and private keys, respectively.

        \STATE \ac{P1} checks the validity of \ac{P2}'s public keys by verifying that $\beta_0 \bullet \beta_1 = C$.  If not, \ac{P1} aborts.

        \STATE \ac{P1} selects $y_0, y_1$ such that $0 \leq y_0, y_1 \leq q-2$, and sends \ac{P2} $a_0 = g^{y_0}, a_1 = g^{y_1}$.
        \STATE \ac{P1} also generates $z_0 = \beta^{y_0}_0, z_1 = \beta^{y_1}_1$ and sends \ac{P2} $r_0 = s_0 \oplus z_0$ and $r_1 = s_1 \oplus z_1$.
        \STATE \ac{P2} computes $z_i = a^{xi}_i$ and then receives $s_i$ by computing $s_i = z_i \oplus r_i$.
    \end{algorithmic}
\end{algorithm}

Step 5 is not in mentioned in the original\cite{bellare1990non} statement of the protocol, but was added by the author to better explain how the protocol works. Here, note that \ac{P1} checks that $\beta_0 \bullet \beta_1 = C$ to prevent \ac{P2} from being able to decrypt under both $\beta_0$ and $\beta_1$, to force \ac{P2} to choose one or the other. As long as the assumption that \ac{P2} does not know the discrete log of $C$ holds, then it follows that \ac{P2} cannot know the discrete log of both $\beta_0$ and $\beta_1$.

\subsection{Securing Circuit Construction}
