Secure function evaluation, or how two parties can jointly compute a function while keeping their inputs private, is an active field in cryptography. In 1986 Andrew Yao presented a solution to the problem called \emph{garbled circuits}, based on modeling the problem as a series of binary gates and encrypting the result tables. This approach was initial treated as theoretically interesting but too computationally expensive for practical use.  However, in the decades since Yao's published his solution, a great deal of work has gone into both optimizing the protocol for practical use, and further securing the protocol to make it useful in untrusted scenarios.

This paper provides a thorough explanation of both Yao's original protocol and its security characteristics.  The paper then details additions to the protocol both to secure it against untrusted parties and to make it practical for computation.  Implementations of Yao's protocol are also discussed, though the paper's emphasis is on the underlying enabling improvements to the protocol.
