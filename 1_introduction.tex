\section{Introduction}
\label{sec:intro}

\ac{SFE} referrers to the problem of how two parties can collaborate to correctly compute the output of a function without either party needing to reveal their inputs to the function, either to each other or to a third party.  A common example of this problem is the ``millionaires' problem'', in which two millionaires want to determine who is richer, without either party revealing how much money they have\cite{yao1982protocols}.

Many solutions have been proposed for \ac{SFE}. One category of solution is function specific, and depends on specific properties of the function being executed to provide security\cite{huang2011faster}.  These solutions, while interesting, are of less general interest, since they apply to only a limited set of problems.

Another category of solution is general in approach, and seeks to provide a general solution for \ac{SFE} by transforming arbitrary functions into secure functions. Approaches in this category include homomorphic encryption systems\cite{gentry2009fully} which allow for arbitrary computation on encrypted data.  Yao's \emph{garbled circuits} protocol also fits in this second, general category.

Yao's \ac{GCP} transforms any function into a function that can be evaluated securely by modeling the function as a boolean circuit, and then masking the inputs and outputs of each gate so that the party executing the function cannot discern any information about the inputs or intermediate values to the function. The protocol is secure as long as both parties do not deviate. A full description of the protocol and the related security definitions are provided later in this paper.

\subsection{History of Protocol}

Interestingly, Yao never published his \ac{GCP}. Several of his publications discuss approaches to the \ac{SFE} problem generally, specifically papers from 1982\cite{yao1982protocols} and 1986\cite{yao1986generate}. These papers are broad and theoretical, and do not directly provide a protocol that could be implemented. Yao first discussed the \emph{garbled circuits} approach in a public talk on the latter paper, as a concrete example of how his broader strategies could be applied\cite{bellare2012foundations}. Only later and by other researchers would the protocol be documented formally\cite{goldreich1987play}, though still crediting Yao for the approach.

That Yao developed this foundational protocol, but never published it, presents writers with the tricky question of what to cite when crediting the \ac{GCP} approach.  The common convention seems to be to cite Yao's two papers discussing his general approach the problem, even though those papers make no mention of garbled circuits.

\subsection{Aims of the Paper}

This paper aims to provide a full description of Yao's \ac{GCP} and its security characteristics.  This paper also provides detailed explanations of related work  by other researchers to improve the performance and security of the protocol.

This paper presumes no previous familiarity with cryptography (generally) or Yao's protocol (specifically) in the sections explaining the protocol, though the general concepts of symmetric and public key cryptography are referenced.  Some background in cryptography is assumed in the sections on security and performance improvements to the protocol.  Formal proofs of the underlying concepts are not discussed and are left to their original papers.

Some discussion is included of existing implementations of Yao's protocol. However, the focus here is on the promises, improvements and general techniques of the implementations, and not on implementation specific details, like implementing programming languages or hardware characteristics. Discussion of the implementations is mainly meant to inform how the protocol has developed and been improved, as opposed to a detailed comparison of how different implementations compare with each other.
\
\subsection{Organization of the Paper}

The remainder of the paper is structured as follows. Section \ref{sec:definitions} provides some security definitions used throughout the rest of the paper. Section \ref{sec:ot} discusses \ac{OT}, its role in the protocol, and a method for achieving \ac{OT} in a manner that is compatible with the security guarantees of the standard version of Yao's protocol.  Section \ref{sec:protocol} then provides a full explanation of the Yao's protocol and how to use \emph{garbled circuits} to solve the \emph{SFE} problem. Section \ref{sec:security} discusses the security of the protocol and proposed improvements, and section \ref{sec:performance} provides a similar discussion of the performance of Yao's protocol. Section \ref{sec:implementations} briefly describes some implementations of the protocol, and section \ref{sec:conclusion} concludes.
