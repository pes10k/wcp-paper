\section{Introduction}

\ac{SFE} referrers to the problem of how can two parties collaborate to correctly compute the output of a function without any party needing to reveal their inputs to the function, either to each other or to a third party.  A common example of this problem is the ``millionaires problem'', in which two millionaires wish to determine which of them has more money, without either party revealing how much money they have\cite{yao1982protocols}.

Many solutions have been developed for \ac{SFE}. One category of solution is function specific, and depends on specific attributes of the function being executed to provide security\cite{huang2011faster}.  These solutions, while interesting, are by definition of less general interest, since they apply to only a limited set of problems.

Another category of approach is more general, and seeks to provide a general solution for \ac{SFE} by transforming arbitrary functions into secure functions. Approaches in this category include homomorphic encryption systems\cite{gentry2009fully} which allow for arbitrary execution on encrypted data.  Yao's \emph{garbled circuits} protocol fits in this second category.

Yao's \ac{GCP} transforms any function into a function that can be evaluated securely by modeling the function as a boolean circuit, and then encrypting the inputs and outputs of each gate so that the party executing the function cannot discern any information about the inputs or intermediate values of the function. The protocol is secure as long as both parties follow the protocol. A full description of the protocol and the related security definitions are provided later in this paper.

\subsection{History of Protocol}

Interestingly, Yao never published his \ac{GCP}. Several of his publications discuss approaches to the \ac{SFE} problem generally, specifically papers from 1982\cite{yao1982protocols} and 1986\cite{yao1986generate}. These papers are much broader in scope and are much more abstract than providing a protocol that could be implemented. Yao first discussed the \emph{garbled circuits} approach in a public talk on the latter paper, as a concrete example of how his broader strategies could be applied\cite{bellare2012foundations}. Only later and by other researchers would the protocol be documented formally\cite{goldreich1987play}, though still crediting Yao for the approach.

Yao having developed this foundational protocol, but never having published it, presents authors with the tricky question of what to cite when crediting to the \ac{GCP} approach.  The common approach seems to be to cite Yao's two papers discussing his general approach the problem, even though those papers make no mention of garbled circuits or any similar concept.

\subsection{Aims of the Paper}

This paper aims to provide a full description of Yao's \ac{GCP} and its security characteristics, namely what security the protocol does and does not provide.  This paper also provides detailed explanations of related work done by other authors to improve the performance and security provided by the protocol.

This paper presumes no previous familiarity with Yao's protocol or cryptography in general in the explanation explanation of the protocol, beyond the general concepts of symmetric and asymmetric cryptography.  Some background in cryptography is assumed in the sections on improvements and additions to the protocol.  Formal proofs of the underlying concepts are not discussed and are left to their originating papers.

Some discussion is included of existing implementations of Yao's protocol. However, the focus here is on the promises, improvements and general techniques of the implementations, and not on implementation details like programming languages or hardware characteristics. Discussion of the implementations is mainly meant to inform how the protocol has developed and been improved, as opposed to a detailed comparison of how different implementations compare with each other.

\subsection{Organization of the Paper}

The remainder of the paper is structured as follows. Section 2 provides some security definitions used throughout the rest of the paper. Section 3 discusses \ac{OT}, its role in the protocol, and a method for achieving \ac{OT} in a manner that is compatible with the security guarantees of the standard version of Yao's protocol.  Section 4 then provides a full explanation of the Yao's protocol and how to use \emph{garbled circuits} to solve the \emph{SFE} problem. Section 5 discusses some of the security and performance characteristics of Yao's protocol, and sections 6 and 7, respectively, discuss subsequent developments to Yao's protocol to address these issues. Section 8 provides a brief overview of some implementations of the protocol, and section 9 concludes.
