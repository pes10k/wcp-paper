\section{Oblivious Transfer}
\label{sec:ot}

\ac{OT} refers to methods for two parties to exchange one-out-of-several values, with the sending party blinded to what value was selected, and the receiving party blinded to all other possible values that could have, but were not, selected.

While \ac{OT} and \ac{SFE} are approaches to distinct (though related) problems, understanding Yao's \ac{GCP} and its security properties requires some understanding of \ac{OT}. Its a cryptographic primitive and a building block that Yao's \ac{GCP} builds on.  This section provides a brief overview of the \ac{OT} problem and a simple \ac{OT} protocol that is secure against \emph{semi-honest} adversaries. The role of \ac{OT} in Yao's protocol is discussed in section \ref{sec:protocol}.

\subsection{Problem Definition}

A general form of \ac{OT} is \emph{1-out-of-N oblivious transfer}, a two party protocol where \ac{P1}, the sending party, has a collection of values. \ac{P2} is able to select one of the values from this set to receive, but is not able to learn any of the other values.

More formally, a \emph{1-out-of-N oblivious transfer} protocol takes as inputs a set of values \emph{N} from \ac{P1}, and index \emph{i} from \ac{P2}, where
$0 \leq i < |N|$. The protocol then outputs nothing to \ac{P1}, and $N_i$ to \ac{P2} in a manner that prevents \ac{P2} from learning $N_j$ for all values of $j \neq i$.

A special case of the above is the \emph{1-out-of-2 oblivious transfer} problem, where \emph{N} is fixed at 2.  Here \ac{P1} has just two values, and \ac{P2} is accordingly limited to $i \in \{0, 1\}$.  All versions of Yao's \ac{GCP} discussed in this paper rely on \emph{1-out-of-2 oblivious transfer} protocols.

\subsection{Example 1-out-of-2 Protocol}

The problem of \emph{1-out-of-2 \ac{OT}} was first addressed by Rabin\cite{rabin2005exchange} in 1981 using an interactive approach with multiple rounds of message passing, but was later adapted into an offline approaches using an techniques similar to the Diffie-Hellman key exchange protocol\cite{diffie1976new}.

The following protocol\cite{lindell2013technion} is a simple \emph{1-out-of-2 \ac{OT}} protocol that is secure against \emph{semi-honest} adversaries. It is included here to assist in the next section's explanation of how the full \ac{GCP} works, and to provide a easy-to-understand example of \ac{OT} to build from later.

\begin{algorithm}[H]
    \floatname{algorithm}{Protocol}
    \caption{Semi-Honest 1-out-of-2 Oblivious Transfer}
    \label{alg:otsemihonest}
    \begin{algorithmic}[1]
        \STATE \ac{P1} has a set of two strings, $S = \{s_0, s_1\}$.
        \STATE \ac{P2} selects $i \in \{0, 1\}$ corresponding to whether she wishes to learn $s_0$ or $s_1$.
        \STATE \ac{P2} generates a public / private key pair $(k^{pub}, k^{pri})$, along with a second value $k^\bot$ that is indistinguishable from a public key, but for which \ac{P2} has no corresponding private key to decrypt with.
        \STATE \ac{P2} then advertises these values as public keys $(k^{pub}_0, k^{pub}_1)$ and sets $k^{pub}_i = k^{pub}$ and $k^{pub}_{i-1} = k^\bot$.
        \STATE \ac{P1} generates $c_0 = E_{k^{pub}_0}(s_0)$ and $c_1 = E_{k^{pub}_1}(s_1)$ and sends $c_0$ and $c_1$ to \ac{P2}.
        \STATE \ac{P2} computes $s_i = D_{k^{pri}}(c_i)$.
    \end{algorithmic}
\end{algorithm}

Note that the protocol is secure by the \emph{semi-honest} definition. As long as no party deviates from the protocol, \ac{P2} is able to recover the desired string $s_i$ but is not able to recover the other value, $s_{i-1}$. Similarly, \ac{P1} never learns $i$.
