\section{Oblivious Transfer}

\ac{OT} refers to methods for two parties to exchange 1 of several values, with the sending party blinded to what value was selected, and the receiving party blinding to all other possible values that could have, but were not selected.

While \ac{OT} and \ac{SFE} are approaches to distinct (though related) problems, understanding the Yao's \ac{GCP} and the security properties requires some understanding of \ac{OT} and how it is used in the protocol. Its a cryptographic primitive that serves as a building block that the security of Yao's \ac{GCP} relies on.

This section provides a brief overview of the \ac{OT} problem, a simple protocol that provides a solution to the \ac{OT} problem against \emph{semi-honest} adversaries. The role of \ac{OT} in Yao's protocol is discussed in section 4.

\subsection{Problem Definition}

A general form of \ac{OT} is \emph{1-out-of-N oblivious transfer}, a two party protocol where \ac{P1}, the sending party, has a collection of values. \ac{P2} is able to select one of the values from this set to receive, but is not able to learn any of the other values.

More formally, a \emph{1-out-of-N oblivious transfer} protocol takes as inputs a set of values \emph{N} from \ac{P1}, and \emph{i} form \ac{P2}, where
$0 \leq i < |N|$. The protocol then outputs nothing to \ac{P1}, and $N_i$ to \ac{P2} in a manner that prevents \ac{P2} from learning another other values in \emph{N}.

A special case of the above is the \emph{1-out-of-2 oblivious transfer} problem, where \emph{N} is fixed at 2.  Here \ac{P1} has just two values, and \ac{P2} is accordingly limited to $i \in \{0, 1\}$.  All versions of Yao's \ac{GCP} discussed in this paper rely on \emph{1-out-of-2 oblivious transfer} protocols.

\subsection{Example 1-out-of-2 Protocol}

The problem of \emph{1-out-of-2 \ac{OT}} was first addressed by Rabin\cite{rabin2005exchange} in 1981 using an online approach with multiple rounds of message passing, but was later adapted into an offline approaches using an techniques similar to the Diffie-Hellman key exchange protocol\cite{diffie1976new}.

The following protocol\cite{lindell2013technion} is a very simple \emph{1-out-of-2 \ac{OT}} protocol that is secure against \emph{semi-honest}. It is included here to help in the next section's explanation of how the full \ac{GCP} works, and to provide a easy-to-understand example of \ac{OT} to build from later.

\begin{framed}
    \noindent\textbf{Semi-Honest 1-out-of-2 Oblivious Transfer}
    \begin{enumerate}
        \item \ac{P1} has a set of two strings, $S = \{s_0, s_1\}$.
        \item \ac{P2} selects $i \in \{0, 1\}$ corresponding to whether she wishes to learn the first or second value in \emph{S}.
        \item \ac{P2} generates two pairs of public / private keys,\\
        $(k^{pub}_0, k^{pri}_0), (k^{pub}_1, k^{pri}_1)$.
        \item \ac{P2} then destroys $k^{pri}_{i-1}$, preventing her from recovering any information encrypted under $k^{pub}_{i-1}$.
        \item \ac{P1} generates $c_0 = E_{k^{pub}_0}(s_0)$ and $c_1 = E_{k^{pub}_1}(s_1)$ and sends $c_0$ and $c_1$ to \ac{P2}.
        \item \ac{P2} computes $s_i = D_{k^{pri}_i}(c_i)$.
    \end{enumerate}
\end{framed}

Note that that as long as all parties follow the protocol, the protocol is secure by the \emph{semi-honest} definition. \ac{P2} is able to recover the desired string from \emph{S} but is not able to recover the other value from \emph{S}. Similarly, \ac{P1} does not know which value from \emph{S} \ac{P2} learned.

% \cite{lindell2013technion}
% 3. Oblivious transfer
%     - Role in Yao in security extensions
%     - distinct concepts, but related (ie Yao relies on some Oblivious transfer method to function)
%         - 1-out-of-N
%         - Yao relies on special case 1-out-of-2
%     - First formalized by Rabin with an online protocol
%         * Rabin, How to Exchange Secrets with Oblivious Transfer [1981]
%     - Simple implementation (semi-honest)
%         * Mihir, Micali, Non-Interactive Oblivious Transfer and Applications
