\section{Implementations}
\label{sec:implementations}

This section briefly describes three significant implementations of Yao's protocol and some of their relevant security and performance characteristics. Other significant and important implementations of the protocol are noted\cite{nielsen2009lego, shen2011two}, but because of time and space constraints are not discussed further.

\subsection{Fairplay}

\emph{Fairplay}\cite{malkhi2004fairplay} was developed in 2004 and was one of the first attempts to create a practical system to execute Yao's protocol.  The system included a high level language for describing functions, which the system would transform into circuit representations suitable for garbling.

The system provided some security against \emph{malicious} adversaries.  For example, a simple cut-and-choose system, similar to the one discussed in section \ref{sec:cutchoosesimple} was included to provide \ac{P2} $1 - 1/m$ protection against corrupt circuits from \ac{P1}.  However, the system is not secure against other attacks from \ac{P1}, such as the corrupt input attack discussed in section \ref{sub:inputattacks}.

\emph{Fairplay} was the first to implement one of the performance techniques discussed in section \ref{sec:performance}, \emph{fast table lookups}.  The system also implemented several different \ac{OT} protocols and found that there were significant performance differences between them.

 \emph{Fairplay} was evaluated against four functions, an AND bitwise operation, the ``billionaires problem'', a simple key database search, and finding the median value in the combined array of the inputs of both parties.  The largest computed circuit consisted of 4383 gates and the system computed these functions on the order of seconds.  Interestingly, the researchers found that communication costs dominated computation costs in all functions. Given the simple nature of most of the problems, \emph{Fairplay} could be said to have shown that Yao's protocol was closer to practical usage than most researchers probably expected, even if it was not there yet. \emph{Fairplay} became the baseline that other implementations were judged against.

\subsection{Huang, Evans, Katz, Malka}

In 2011 Huang, et al.\cite{huang2011faster} presented a new practical system for carrying out Yao's protocol.  Like \emph{Fairplay}, the system used a high level language that users could write functions in, in this case Java.  The system would then automatically translate these high level functions into circuit representations.

This approach focused on the \emph{semi-honest} scenario, and thus made the trade off of less security for faster performance (the paper explicitly discusses the reasoning behind this decision). The system included the performance optimizations in \emph{Fairplay}, but added many others, including free XORs (discussed in section \ref{sec:freexoropt}), garbled gate reduction (discussed in section \ref{sec:grr}) and pipelined execution (discussed in section \ref{sec:piplinedexecution}).

The system was compared against previously best known privacy preserving methods for computing Hamming distance, Levenshtein distance, Smith-Waterman genome alignment and AES, and found an order of magnitude improvement in their system (though it is not discussed if the compared to approaches are secure in the \emph{malicious} case or similarly just in the \emph{semi-honest} case). The largest circuit computed consisted of over 1 billion gates, for the Levenshtein distance problem.

\pagebreak
\subsection{Kreuter, Shelat, Shen}

Most recently, Kreuter et al.\cite{kreuter2012billion} presented a system for carrying out Yao's protocol in the \emph{malicious} case.  Similar to the previously discussed methods, this system includes a way of converting from a high level representation of a function into a boolean circuit.  Here the high level language was created by the authors.  The compiler is also specifically compared to the \emph{Fairplay} complier.  The authors find that their boolean circuit compiler is able to generate dramatically larger circuits using fewer computational resources than previous efforts.

This work represents the state of the art in securing Yao's protocol in the \emph{malicious} case.  While the work is not quite able to claim full security against \emph{malicious} adversaries (because of the open problem of \ac{P2} being able to abort early and not reveal the circuit's output), it appears to be as close as possible given known methods.  The work includes all security techniques discussed in section \ref{sec:security}, as well as some other strategies that are beyond the scope of this paper.

Similarly, this work also includes all discussed performance optimizations discussed in section \ref{sec:performance}. It also incorporates strategies that are not discussed in this paper because they are either hardware dependent (the system uses hardware optimizations in the \emph{Intel Advanced Encryption Standard Standard Instructions} provided on some Intel and AMD processors) or highly heuristic based and possibly not generally applicable (see the system's method for limiting the working set of a garbled circuit).

The system was evaluated on AES, RSA signing, dot product computation, and edit distance of large strings.  The evaluations show that the random seed checking (section \ref{sec:randomseed}), garbled row reduction (section \ref{sec:grr}) and free XOR (section \ref{sec:freexoropt}) optimizations provide the greatest improvement.  The largest circuit computed in the system was in finding the edit distance between two 4095 bit strings, which required over 5.9 billion gates and 8.2 hours.  This is compared to the Huang et al. approach to the same problem, which computed the same edit distance computation more quickly, but in the \emph{semi-honest} setting.
